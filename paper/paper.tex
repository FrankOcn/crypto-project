\documentclass{article}
\usepackage[american]{babel}
\usepackage{csquotes}
\usepackage[backend=biber,bibstyle=mla]{biblatex}
\usepackage{mathtools}
\DeclarePairedDelimiter{\ceil}{\lceil}{\rceil}
\addbibresource{references.bib}
\author{Cristian Vergara}
\title{Breaking Diffie-Hellman}
\date{\today{}}
\begin{document}
  \maketitle{}
  \tableofcontents{}
  \newpage{}
  \section{The Problem}
    \begin{displayquote}
      Two persons (no doubt evil) are employing a Diffie-Hellman scheme to fix
      up a session key for their (no doubt evil) communications. They are
      using a prime modulus of

          3217014639198601771090467299986349868436393574029172456674199

          and a base value of  5.

          Your operatives report that they have witnessed the Diffie-Hellman
      exhanges of the following information

          244241057144443665472449725715508406620552440771362355600491

          and

          794175985233932171488379184551301257494458499937502344155004.

          Find the shared secret key.

      Good luck - and have fun,

      KDB
    \end{displayquote}
    \subsection{A few immediate obersations}
      The prime modulus $p$ given to use is a 202-bit number, or 61 digits in
      base 10.

      We are given two values: $g^x$ and $g^y$, and we're asked to find
      $g^{xy}$. This means that we only need to solve for either $x$ or $y$,
      because $(g^x)^y = (g^y)^x = g^{xy}$.

      There's a lot more to be said about this problem, but the discoveries that
      we made will be explained where appropriate.
  \section{Baby-Step Giant-Step}
    The Baby-Step Giant-Step algorithm is a meet-in-the-middle attack on the
    discrete log problem.

    Given $g$, $b$ and $p$, we want to find $x$ in $g^x \equiv b \pmod{p}$.
    Instead, we express $x$ as $im + j$ with $m = \ceil{\sqrt{n}}$.
  \newpage{}
  \printbibliography
\end{document}
